\documentclass[11pt]{article}
\usepackage{amsmath, amsfonts, amssymb, graphicx}
\usepackage[round]{natbib}
\usepackage{dsfont}
%\usepackage{cleveref}
\newcommand{\given}{\,\vert\,}
\renewcommand\bibsection{\subsubsection*{\center \sc References}}

\pagestyle{plain}
\oddsidemargin 0in
\evensidemargin 0in
\marginparwidth 0in
\marginparsep 0in
\topmargin 0in
\headheight 0in
\headsep 0in
\textheight 9in
\textwidth 6.5in
\raggedbottom
\linespread{1.6}

%% Some commands to keep the sysbio.bst from generating weird errors
\newcommand{\bibAnnoteFile}[1]{
\IfFileExists{#1}{\begin{quotation}\noindent\textsc{Key:} #1\\
\textsc{Annotation:}\ \input{#1}\end{quotation}}{}}

\newcommand{\bibAnnote}[2]{%
\begin{quotation}\noindent\textsc{Key:} #1\\
\textsc{Annotation:}\ #2\end{quotation}}

\begin{document}
\bibliographystyle{sysbio}
\bibpunct[; ]{(}{)}{;}{a}{}{;}

\begin{titlepage}
\begin{center}
{\Large\bf Phylowood: Interactive Web-based Animations of\\Biogeographical and Phylogeographical Histories}

\vfill

R.H. Phylowood Animations

\vfill

{\sc Michael J. Landis$^{\,1}$ and Trevor Bedford$^{\,2}$} \\

\bigskip

{\em
$\mbox{}^1$Department of Integrative Biology\\
\vspace{-0.4\baselineskip}
University of California, Berkeley, CA 94720-3140, \mbox{U.S.A.} \\

$\mbox{}^2$Institute of Evolutionary Biology\\
\vspace{-0.4\baselineskip}
University of Edinburgh, Edinburgh, UK \\
}
\end{center}

\vfill

\begin{flushleft}
Michael J. Landis \\
\vspace{-0.4\baselineskip}
University of California, Berkeley \\ 
\vspace{-0.4\baselineskip}
Department of Integrative Biology \\
\vspace{-0.4\baselineskip}
3060 VLSB \#3140 \\
\vspace{-0.4\baselineskip}
Berkeley, CA 94720-3140 \\
\vspace{-0.4\baselineskip}
\mbox{U.S.A.}

Phone: (510) 642-1233 \\
\vspace{-0.4\baselineskip}
E-mail: {\tt mlandis@berkeley.edu} \\
\end{flushleft}

\end{titlepage}

\newpage

\begin{center}
{\sc Abstract}
\end{center}

\noindent {\it Summary}.---Phylowood is a web service that generates animations of biogeographical and phylogeographical histories. The animations are interactive, allowing the user to adjust spatial and temporal resolution, and highlight phylogenetic lineages of interest.

\noindent {\it Availability}.---\texttt{http://mlandis.github.com/phylowood}

\bigskip

\noindent [ancestral area analysis; historical biogeography; phylogeography; interactive animation; web service]

\begin{center}
{\sc Introduction}
\end{center}

The fields of phylogeography and biogeography study the processes that gave rise to the geographical distributions of populations and species, respectively. Methods to reconstruct ancestral species and population distributions in a phylogenetic context have recently enjoyed increased popularity (\citep{ronquist97, ree08, lemmon08, lemey09, yu10, landis12}. The resulting ancestral distribution reconstructions are inherently high dimensional, since all lineages have inferred measurements that vary with respect to time and space, and can be difficult to interpret. We introduce Phylowood, a web utility that generates interactive animations to facilitate the exploration and summarization of such complex reconstructions.

\bigskip

\begin{center}
{\sc Use}
\end{center}

Phylowood has two primary display panels: the phylogeny panel and the geography panel (Figure \ref{phylowood}). The phylogeny panel (left) contains a time-calibrated phylogeny, where lineages are assigned unique colors by which the ancestral states contained in the geography panel (right) are identified. Below the tree, standard media buttons control the animation speed, direction, and location. The animation time slider indicates the lineages that exist and whose histories are being animated for that point in time.

[Figure 1]

Filtering out uninteresting data is key to exploration and summarization. Phylowood allows users to mask, unmask, and highlight sets of branches using simple doubleclick, click, and mouseover events through either the phylogeny or geography displays. For example, mask and unmask commands may be used to remove all but 10 lineages from a dataset containing 1000 taxa. To unmask all lineages, simply single click the branch leading the root. Mouseover events provide information about highlighted lineage below the geography panel.

The geographical panel contains a dynamic map, capable of zooming and panning, and the area markers corresponding to the animation time slider for unmasked phylogenetic lineages. So animations reflect the underlying model assumptions of the ancestral area reconstructions, we allow several animation styles: continuous phylogeography, discrete phylogeography, and discrete biogeography. At a given node, the area (size) of a marker is proportional to the measurement for that lineage at that location (e.g. posterior probability, confidence metric, parsimony score). Intermediate values along lineages are animated using interpolated ancestral area values (continuous phylogeographical models also interpolate the geographical coordinates). Similar to the branches in the phylogenetic panel, area markers respond to the highlight mouseover command.

While we provide several demonstrative datasets, users can easily animate their own results through the web service. To further explore Phylowood's features, visit \texttt{http://github.com/mlandis/phylowood/wiki} for help and tutorials.

\bigskip

\begin{center}
{\sc Implementation}
\end{center}

Phylowood was developed in Javascript, and thus can be used with any HTML5-compliant web browsers regardless of operating system. SVG objects are managed and animated using D3.js (\texttt{http://d3js.org/}). Map tiles are fetched from Cloudmade \texttt{http://cloudmade.com/} using Polymaps \texttt{http://polymaps.org/}. Source code is is published GNU Software License and made freely available at \texttt{http://github.com/mlandis/phylowood}.

Animations are generated from a NEXUS format file, specifying the animation settings, the geographical coordinates, the taxa names, and a New Hampshire eXtended format tree annotated with ancestral area values. This format was adopted to accommodate a variety of methods, since Phylowood is agnostic about the method underlying the results. We currently provide Ruby scripts to convert BEAST output to Phylowood format, with more to be developed upon demand. BayArea produces Phylowood format files natively.

\bigskip

\begin{center}
{\sc Acknowledgments}
\end{center}

This research was supported by funding provided by Google Summer of Code 2012 awarded to M.J.L., grants from the NSF (DEB-0445453) and NIH (GM-069801) awarded to John P. Huelsenbeck, and [ ?? funding sources for TB ?? ].

\bibliography{bayes}

\newpage

%%%%%%%%%
% FIGURES
%%%%%%%%%
\newpage

\begin{figure}
\begin{center}
\caption{
Sample still frame from Phylowood. The results shown are from the discrete biogeographical analysis of {\it Rhododendron} section {\it Vireya} in Malesia using BayArea. Much of the phylogenetic tree has been masked using mouse-issue commands, leaving two clades and their shared ancestry unmasked. The media slider indicates the current position of the animation with respect to the time-calibrated phylogeny, for which time six unmasked lineages are animated.  For the current animation time, each extant lineage is allocated an equal width slice of the pie. For each color, the depth of the slice indicates the approximate marginal posterior probability of the lineage occupying the area at that time. Pie slices are sorted phylogenetically, making the relative position of absent slices informative. Consulting the geography panel, we find the taxa from the top clade appear to be allopatric with respect to taxa from the bottom clade at -15.5Myr. The interactive animation is available at \texttt{http://mlandis.github.com/phylowood/?url=examples/vireya.nhx}.
}
\includegraphics[width=6.50in]{phylowood}
\label{phylowood}
\end{center}
\end{figure}

\end{document}


 

