\documentclass[11pt]{article}
\usepackage{amsmath, amsfonts, amssymb, graphicx}
\usepackage[round]{natbib}
\usepackage{dsfont}
%\usepackage{cleveref}
\newcommand{\given}{\,\vert\,}
\renewcommand\bibsection{\subsubsection*{\center \sc References}}
\usepackage[labelfont=bf,labelsep=period,font=small]{caption}

\pagestyle{plain}
\oddsidemargin 0in
\evensidemargin 0in
\marginparwidth 0in
\marginparsep 0in
\topmargin 0in
\headheight 0in
\headsep 0in
\textheight 9in
\textwidth 6.5in
\raggedbottom
\linespread{1.6}

%% Some commands to keep the sysbio.bst from generating weird errors
\newcommand{\bibAnnoteFile}[1]{
\IfFileExists{#1}{\begin{quotation}\noindent\textsc{Key:} #1\\
\textsc{Annotation:}\ \input{#1}\end{quotation}}{}}

\newcommand{\bibAnnote}[2]{%
\begin{quotation}\noindent\textsc{Key:} #1\\
\textsc{Annotation:}\ #2\end{quotation}}

\begin{document}
\bibliographystyle{sysbio}
\bibpunct[; ]{(}{)}{;}{a}{}{;}

\begin{titlepage}
\begin{center}
{\Large\bf Phylowood: Interactive Web-based Animations of\\Biogeographical and Phylogeographical Histories}
% R.H. Phylowood Animations

\bigskip

{\sc Michael J. Landis$^{\,1}$ and Trevor Bedford$^{\,2}$} \\

\bigskip

{\em
$\mbox{}^1$Department of Integrative Biology\\
\vspace{-0.4\baselineskip}
University of California, Berkeley, CA 94720-3140, \mbox{U.S.A.} \\

$\mbox{}^2$Institute of Evolutionary Biology\\
\vspace{-0.4\baselineskip}
University of Edinburgh, Edinburgh, UK \\
}
\end{center}

\bigskip

\begin{abstract}

\noindent {\bf Summary:} Phylowood is a web service that uses Javascript to generate in-browser animations of biogeographical and phylogeographical histories from annotated phylogenetic input. The animations are interactive, allowing the user to adjust spatial and temporal resolution, and highlight phylogenetic lineages of interest.

\noindent {\bf Availability:} All documentation and source code for Phylowood is freely available at \\
\texttt{https://github.com/mlandis/phylowood} and a live web application is available at \\
\texttt{https://mlandis.github.io/phylowood}.

\noindent {\bf Contact:} mlandis@berkeley.edu

\end{abstract}

\end{titlepage}



% \noindent [ancestral area analysis; historical biogeography; phylogeography; interactive animation; web service]

\section{Introduction}

The fields of phylogeography and biogeography study the processes that give rise to the observed geographical distributions across individuals and across species, respectively. Methods to infer migration processes and reconstruct ancestral geographical distributions in a phylogenetic context have recently enjoyed increased popularity \citep{ronquist97, beerli01, ree08, lemmon08, lemey09, yu10, landis12}. The resulting ancestral reconstructions are inherently high dimensional; they describe distributions across space, time and phylogenetic lineage, and consequently, can be difficult to interpret. Here, we introduce Phylowood, a web utility that generates interactive animations to facilitate the exploration and summarization of such complex reconstructions.

Phylowood takes phylogeographic output from BEAST \citep{drummond12} in the form of NEWICK trees in which internal nodes are annotated with inferred ancestral state or internal branches are annotated with migration events. Phylogeographic state can be discrete or continuous. Additionally, Phylowood takes biogeographic output from BayArea [CITE] in a similar annotated NEWICK format.  Biogeographic reconstructions differ in associating species with multiple geographic locations representing the species' range, instead of associating an individual with a single geographic location. Fundamentally, Phylowood plots the reconstructed geographic distribution across individuals or species at a particular point in time and explores temporal dynamics through animation.

\section{Use}

Phylowood has two primary display panels: the phylogeny panel and the geography panel (Figure \ref{phylowood}). The phylogeny panel (left) contains a time-calibrated phylogeny, where lineages are assigned unique colors that reflect phylogenetic proximity. The geography panel (right) contains colored area markers corresponding to phylogenetic lineages that specify discrete or continuous geographic distributions. Below the tree, standard media buttons control the animation speed, direction, and location. The animation time slider shows the current time point and indicates the lineages that exist at this time, which comprise targets for animation.

\begin{center}
[Figure 1]
\end{center}

Filtering out uninteresting data is key to exploration and summarization. Phylowood allows users to mask, unmask, and highlight sets of branches using simple doubleclick, click, and mouseover events through either the phylogeny or geography displays. For example, mask and unmask commands may be used to remove all but 10 lineages from a dataset containing 1000 taxa. To unmask all lineages, simply single click the branch leading the root. Mouseover events provide information about highlighted lineage below the geography panel and help users to connect phylogenetic lineages to their geographic counterparts.

The geography panel contains a dynamic map, capable of zooming and panning, and the area markers representing geographic distributions at the current time point for unmasked phylogenetic lineages. As geographic representations reflect the underlying model assumptions of the ancestral area reconstructions, we allow several styles: continuous phylogeography, discrete phylogeography, and discrete biogeography. In the case of continous phylogeographic state, each node in the reconstructed phylogeny has a unique latitude and longitude. Here, Phylowood displays each branch in the phylogeny as a separate geographic marker. In the case of discrete phylogeography or discrete biogeography, there are a limited number of possible geographic locations to which phylogenetic lineages may be assigned. At a given phylogenetic branch, the area (size) of a corresponding geographic marker is proportional to the weight assigned to that geographic state. Depending on reconstruction methodology, this weight may represent posterior probability, confidence metric, or parsimony score. Intermediate values along phylogenetic branches are interpolated between reconstructed internal phylogenetic nodes. In the discrete phylogeographic or biogeographic scenarios, interpolation represents posterior probability, etc..., of assignment along a branch of the phylogeny. In the continous phylogeographic scenario, interpolation represents the reconstructed continous location along a branch of the phylogeny. Similar to the branches in the phylogenetic panel, area markers respond to the highlight mouseover command.

\section{Implementation}

Phylowood was developed in Javascript, and thus can be used with any HTML5-compliant web browsers regardless of operating system. SVG objects are managed and animated using D3.js (\texttt{http://d3js.org/}). Map tiles are fetched from Cloudmade \texttt{http://cloudmade.com/} using Polymaps \texttt{http://polymaps.org/}. Source code is is published GNU Software License and made freely available at \texttt{http://github.com/mlandis/phylowood}.

Animations are generated from a NEXUS format file, specifying the animation settings, the geographical coordinates, the taxa names, and a New Hampshire extended format (NEWICK) tree annotated with ancestral area values. This format was adopted to accommodate a variety of methods, since Phylowood is agnostic about the method underlying the results. We currently provide Ruby scripts to convert BEAST output to Phylowood format, with more to be developed upon demand. BayArea produces Phylowood format files natively.

While we provide several demonstrative datasets, users can easily animate their own results through the web service. To further explore Phylowood's features, visit \texttt{http://github.com/mlandis/phylowood/wiki} for help and tutorials.

\section{Example}

[WE LEARN SOMETHING ABOUT VIREYA USING BAYAREA + PHYLOWOOD.]

\section{Conclusion}

[IMPORTANCE OF SOPHISTICATED VISUALIZATION TECHNIQUES AS DATA INCREASES]

\section*{Acknowledgments}

MJL was supported by funding provided by Google Summer of Code 2012, and grants from the NSF (DEB-0445453) and NIH (GM-069801) awarded to John P.\ Huelsenbeck.  TB was supported by a Newton International Fellowship from the Royal Society.  

\bibliography{bayes}

\newpage

%%%%%%%%%
% FIGURES
%%%%%%%%%
\newpage

\begin{figure}
\begin{center}
\includegraphics[width=\textwidth]{phylowood}
\caption{
Sample still frame from Phylowood. The results shown are from the discrete biogeographical analysis of {\it Rhododendron} section {\it Vireya} in Malesia using BayArea. Much of the phylogenetic tree has been masked using mouse-issue commands, leaving two clades and their shared ancestry unmasked. The media slider indicates the current position of the animation with respect to the time-calibrated phylogeny, for which time six unmasked lineages are animated.  For the current animation time, each extant lineage is allocated an equal width slice of the pie. For each color, the depth of the slice indicates the approximate marginal posterior probability of the lineage occupying the area at that time. Pie slices are sorted phylogenetically, making the relative position of absent slices informative. Consulting the geography panel, we find the taxa from the top clade appear to be allopatric with respect to taxa from the bottom clade at -15.5Myr. The interactive animation is available at \texttt{http://mlandis.github.io/phylowood/?url=examples/vireya.nhx}.
}
\label{phylowood}
\end{center}
\end{figure}

\end{document}


 

